\documentclass{article}
\usepackage[utf8]{inputenc}
\usepackage{pgfplots}
\pgfplotsset{width=10cm,compat=1.9}
\usepackage{amsmath,amssymb,amsthm}
\usepackage{gensymb}
\usepackage{graphicx}
\usepackage{amsmath}
\usepackage{float}
\usepackage{enumerate}
\usepackage{xcolor}
\usepackage{blindtext}
\usepackage[margin=1in]{geometry}
\usepackage{hyperref}
\hypersetup{
    colorlinks=true,
    linkcolor=blue,
    filecolor=magenta,      
    urlcolor=cyan,
    pdftitle={Overleaf Example},
    pdfpagemode=FullScreen,
    }
\usepackage[slovene]{babel}

\setlength{\parindent}{0pt}
\setlength{\parskip}{4pt}


\newcounter{example}[section]
\newenvironment{example}[1][]{\refstepcounter{example}\par\medskip
   \noindent \textbf{Naloga~\theexample. #1} \rmfamily}{\medskip}


\title{Verjetnost vaje}
\author{Bor Bregant}
\date{\vspace{-5ex}}

\begin{document}

\maketitle

\section{Osnovni nivo}

\begin{example}
    Verjetnost, da je v vseh permutacijah besede SLOVENIJA pojavi beseda LOVE
\end{example}

\begin{example}
    Iz števk 1, 2, 3 sestavljamo 6-mestna števila. Koliko je vseh možnosti. Kolikšna je verjetnost, da bo imelo število števko 1 natanko enkrat in števko 2 vsaj trikrat. \textcolor{gray}{pogledamo koliko je možnosti za 3, 4, 5 dvojk in to seštejemo.}
\end{example}

\begin{example}
    Števke 1, 2, 3, 4, 5, 6 in sestavljamo 4-mestna števila. Kolikšna je verjetnost, da bo izbrano število deljivo s 5.
\end{example}

\begin{example}
    V zaboju 120 žarnic in $20\%$ pregorenih. Kolikšna je verjetnost, da če jih 5 potegnemo ven, da bodo vsaj 4 gorele.
\end{example}

\begin{example}
    7 jih gre v kino v isto vrsto. Kolikošna je verjetnost, da bosta dve sedeli skupaj. \textcolor{gray}{lahko se tudi zamenjata torej $\cdot 2$}
\end{example}

\begin{example}
    8 belih, 12 črnih kroglic. Kolikšna verjetnost, da izvlečemo 2 beli in 2 črni.
\end{example}

\begin{example}
    Višji nivo:  5 naključno izbranih ljudi vprašamo, kdaj imamo rojstni dan. Kolikšna je verjetnost, da ga imajo na različne dneve tedna. \textcolor{gray}{$P(A)=\frac{7\cdot \ldots \cdot 3}{7^5}$}
\end{example}

\begin{example}
    Kolikšna je verjetnost, da je produkt dveh pik na kockah manjši od 6. \textcolor{gray}{Vzorčni prostor}
\end{example}

\begin{example}
    10 črnih, 8 belih. Verjetnost za CC, CC, CB, CB z vračanjem ali brez
\end{example}

\begin{example}
    Kolikšna je verjetnost, da v kompletu 32 kart dobi en igralec vse štiri ase, če si razdelijo po 8 kart.
\end{example}

\end{document}
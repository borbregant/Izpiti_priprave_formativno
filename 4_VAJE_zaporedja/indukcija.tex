\documentclass{article}
\usepackage[utf8]{inputenc}
\usepackage{pgfplots}
\pgfplotsset{width=10cm,compat=1.9}
\usepackage{amsmath,amssymb,amsthm}
\usepackage{gensymb}
\usepackage{graphicx}
\usepackage{amsmath}
\usepackage{float}
\usepackage{enumerate}
\usepackage{xcolor}
\usepackage{blindtext}
\usepackage[margin=1in]{geometry}
\usepackage{hyperref}
\hypersetup{
    colorlinks=true,
    linkcolor=blue,
    filecolor=magenta,      
    urlcolor=cyan,
    pdftitle={Overleaf Example},
    pdfpagemode=FullScreen,
    }
\usepackage[slovene]{babel}

\setlength{\parindent}{0pt}
\setlength{\parskip}{4pt}


\newcounter{example}[section]
\newenvironment{example}[1][]{\refstepcounter{example}\par\medskip
   \noindent \textbf{Naloga~\theexample. #1} \rmfamily}{\medskip}


\title{Indukcija vaje}
\author{Bor Bregant}
\date{\vspace{-5ex}}

\begin{document}

\maketitle

\begin{example}
    Dokaži
    \begin{align*}
    &\sum_{i=1}^{n}i^2=\frac{n(n+1)(2n+1)}{6}\\
    &\frac{1}{1\cdot 2}+\frac{1}{2\cdot 3}+\ldots+\frac{1}{n(n+1)}=\frac{n}{n+1}
    \end{align*}
\end{example}

\begin{example}
    Dokaži
    \begin{align*}
    &3|(n^3+2n)\forall n\in\mathbb{N}\\
    &6|(17n^3+103n)\\
    \end{align*}
\end{example}

\begin{example}
    DN: Dokaži
    \begin{align*}
    &1^3+2^3+3^3+\ldots +n^3=\frac{1}{4}n^2(n+1)^2\\
    &7|(5^{2n+1}+2^{2n+1})
    \end{align*}
\end{example}


\end{document}
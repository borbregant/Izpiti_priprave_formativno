\documentclass{article}
\usepackage[utf8]{inputenc}
\usepackage{pgfplots}
\pgfplotsset{width=10cm,compat=1.9}
\usepackage{amsmath,amssymb,amsthm}
\usepackage{gensymb}
\usepackage{graphicx}
\usepackage{amsmath}
\usepackage{float}
\usepackage{enumerate}
\usepackage{xcolor}
\usepackage{blindtext}
\usepackage[margin=1in]{geometry}
\usepackage{hyperref}
\hypersetup{
    colorlinks=true,
    linkcolor=blue,
    filecolor=magenta,      
    urlcolor=cyan,
    pdftitle={Overleaf Example},
    pdfpagemode=FullScreen,
    }
\usepackage[slovene]{babel}

\setlength{\parindent}{0pt}
\setlength{\parskip}{4pt}


\newcounter{example}[section]
\newenvironment{example}[1][]{\refstepcounter{example}\par\medskip
   \noindent \textbf{Naloga~\theexample. #1} \rmfamily}{\medskip}


\title{Funkcije vaje}
\author{Bor Bregant}
\date{\vspace{-5ex}}

\begin{document}

\maketitle

\section{Osnovni nivo}

\begin{example}
    Zapiši definici
\end{example}

\section{Višji nivo}

\begin{example}
    Za funkcijo $f(x)=\frac{x}{1-x}$ ugani predpis za $(f\circ f \circ \ldots \circ f)(x)$ in ga z indukcijo dokaži
\end{example}

\begin{example}
    Določi definicijsko območje in inverzno funkcijo za $f(x)=\ln \frac{2x-1}{x+2}$.
\end{example}


\end{document}
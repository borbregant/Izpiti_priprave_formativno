\documentclass{article}
\usepackage[utf8]{inputenc}
\usepackage{pgfplots}
\pgfplotsset{width=10cm,compat=1.9}
\usepackage{amsmath,amssymb,amsthm}
\usepackage{gensymb}
\usepackage{graphicx}
\usepackage{amsmath}
\usepackage{float}
\usepackage{enumerate}
\usepackage{xcolor}
\usepackage{blindtext}
\usepackage[margin=1in]{geometry}
\usepackage{hyperref}
\hypersetup{
    colorlinks=true,
    linkcolor=blue,
    filecolor=magenta,      
    urlcolor=cyan,
    pdftitle={Overleaf Example},
    pdfpagemode=FullScreen,
    }
\usepackage[slovene]{babel}

\setlength{\parindent}{0pt}
\setlength{\parskip}{4pt}

\thispagestyle{empty}

\newcounter{example}[section]
\newenvironment{example}[1][]{\par\medskip
   \noindent \textbf{Naloga:} \rmfamily}{\medskip}


\title{Delo v parih}
\author{Bor Bregant}
\date{\vspace{-5ex}}

\begin{document}

\section*{V parih rešite naslednje naloge. Posvetujte se s sošolcem, v primeru nejasnosti pa vprašajte profesorja.}


\begin{example}
    Reši enačbi
    \begin{align*}
        \frac{1}{2x+1}&=\frac{1}{4} \ \text{in}\\
        \frac{2x}{x-1}+\frac{1}{x-3}&=\frac{2}{x^2-4x+3}\\
    \end{align*}
\end{example}

\textcolor{gray}{Rešitev $x=\frac{3}{2}$ in $x_1 = -\frac{1}{2},\, x_2=3,\ \text{ki pa ni rešitev}$}

\begin{example}
    Izračunaj, kje se sekata $f(x)=\frac{x^3+3x+4}{x^2-4}$ in $g(x)=\frac{2}{x-2}$.
\end{example}

\textcolor{gray}{Rešitev $P(0,-1)$}

\begin{example}
    Poišči presečišče $f(x)=\frac{x^2-9}{-x^2+2x-1}$ s premico $x-y+3=0$.
\end{example}

\textcolor{gray}{Rešitev $P_1 (-3,0), P_2(-1,2), P_3(2,5)$}

\section*{V parih rešite naslednje naloge. Posvetujte se s sošolcem, v primeru nejasnosti pa vprašajte profesorja.}


\begin{example}
    Reši enačbi
    \begin{align*}
        \frac{1}{2x+1}&=\frac{1}{4} \ \text{in}\\
        \frac{2x}{x-1}+\frac{1}{x-3}&=\frac{2}{x^2-4x+3}\\
    \end{align*}
\end{example}
    
\textcolor{gray}{Rešitev $x=\frac{3}{2}$ in $x_1 = -\frac{1}{2},\, x_2=3,\ \text{ki pa ni rešitev}$}

\begin{example}
    Izračunaj, kje se sekata $f(x)=\frac{x^3+3x+4}{x^2-4}$ in $g(x)=\frac{2}{x-2}$.
\end{example}

\textcolor{gray}{Rešitev $P(0,-1)$}

\begin{example}
    Poišči presečišče $f(x)=\frac{x^2-9}{-x^2+2x-1}$ s premico $x-y+3=0$.
\end{example}

\textcolor{gray}{Rešitev $P_1 (-3,0), P_2(-1,2), P_3(2,5)$}

\end{document}